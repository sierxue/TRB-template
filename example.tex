\documentclass[12pt]{trbart}

\title{TRB Template: an Example}

\author{Author A}
\author{Author B}
\affil{University First\\ Address First\\ \email{first@example.com}}
\author{Author C}
\affil{University Second\\ Address Second\\ \email{second@example.com}}

\date{}

\usepackage{lipsum}  % For test only

\begin{document}

\linenumbers{}
\maketitle

\section{A section}
\lipsum[1]

\subsection{A subsection}
\lipsum[2]

\begin{equation}\label{eq:exp}
  \lim_{n\to \infty} \left(1 + \frac{1}{n}\right){}^n = e  
\end{equation}

\lipsum[1]

\subsubsection{A cute subsubsection}
\lipsum[1]
\begin{figure}[!hbt]
    \centering
    \includegraphics[width=0.7\textwidth]{example-image-a}
    \caption{This is a figure with a very long caption because I want to show you how the indentation works when there is a line break in a figure's title.\\ I'm not sure about the indent for titles with more than one paragraph.}\label{fig:longtitle}
\end{figure}

\lipsum[1]

\subsection{Another subsection}

\lipsum[1]
\begin{table}[!hbt]
    \centering
    \caption{A table with bold and center-aligned headers}\label{tab:shorttitle}
    \begin{tabularx}{0.5\textwidth}{crEE}
        \toprule
        \multirowthead{2}{Obs} & \multirowthead{2}{Attr 1} & \multicolumn{2}{c}{\thead{Attr 2}} \\
        \cmidrule(lr){3-4}
        & & \thead{SubAttr 1} & \thead{SubAttr 2} \\
        \midrule
        A & 1.0 & 2.0 & -12.0 \\
        B & -3.0 & -101.4 & 4.7 \\
        \bottomrule
    \end{tabularx}
\end{table}

We can refer to equations or floatings using \verb+\trbref+ commands, showing as \autoref{eq:exp}, \autoref{fig:longtitle} and \autoref{tab:shorttitle}.

\section{Last Section}

\lipsum[1]

\end{document}