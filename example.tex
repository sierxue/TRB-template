\documentclass[12pt]{trbart}

\title{TRB Template: an Introduction of `trbart' documentclass}

\author{Author A}
\author{Author B}
\affil{University First\\ Address First\\ \email{first@example.com}}
\author{Author C}
\affil{University Second\\ Address Second\\ \email{second@example.com}}

\date{}

\usepackage{lipsum}  % For test only

\begin{document}

\linenumbers{}
\maketitle


\section{Introduction}
This documents is an example of the \texttt{trbart} class \LaTeX\ file maintained in \url{https://github.com/wklchris/TRB-template}. I would introduce main settings of the template through this document.

Some settings are simple and I just list them here:
\begin{itemize}
    \item Paragraph indentation is \(0.5\) inches. First paragraph of each sections are also indented.
\end{itemize}


\section{Outline Formats}
This template is based on the standard \texttt{article} documnetclass, and there are 3 outline levels in total:
\begin{itemize}
    \item Section: Each letter is captalized. Bold series.
    \item Subsection: Initial letter of each word is captalized. Bold series.
    \item Subsubsection: Only first letter of the whole title is captalized. Italic shape.
\end{itemize}

Here are examples to show the format of different outline levels.

\subsection{A subsection example}
This is a subsection example.

\subsubsection{A subsubsection example}
This is a subsubsection example.


\section{Titlepage}
The titlepage is inserted by the command \verb+\maketitle+. Author names are in bold font, followed by the affiliation of the author, the street address, the city \& state with zipcode, and his/her email.


\section{Equations}
Equations are left-aligned and counted towards line numbers. This style is achieved by \texttt{fleqn} option of the \texttt{amsmath} package with \verb+\mathindent+ set to zero. Please use \texttt{equation} environment to write equations so that they can be correctly linenumbered --- I only configured the compatiblity between this environment and the \texttt{fleqn} option. Let's try a well-known theorem:
\begin{equation}\label{eq:triangle}
    a^2 + b^2 = c^2
\end{equation}
where \(c\) is the hypotenuse of a right triangle, \(a\) and \(b\) are the two legs.

Here is a more complicated example. Though there are two lines, but the this equation will be counted as whole in the line number sequence.
\begin{equation}\label{eq:exp}
\begin{dcases}
    \frac{\ud \boldsymbol{p}}{\ud t} &= -\pfrac{H}{\boldsymbol{q}} \\
    \frac{\ud \boldsymbol{q}}{\ud t} &= \pfrac{H}{\boldsymbol{p}}
\end{dcases} 
\end{equation}

The citation style of equations (\autoref{eq:triangle} and \autoref{eq:exp}) are bold with the default color.

\section{Floatings: Figures and Tables}
There is nothing special for figures. \autoref{fig:longtitle} is an example of a figure with a long title.
\begin{figure}[!hbt]
    \centering
    \includegraphics[width=0.7\textwidth]{example-image-a}
    \caption{This is a figure with a very long caption because I want to show you how the indentation works when there is a line break in a figure's title.\\ I'm not sure about the indent for titles with more than one paragraph.}\label{fig:longtitle}
\end{figure}

You may include table with \texttt{tabularx} package. The \LaTeX\ code below will create an output of \autoref{tab:shorttitle}. Table typesetting varies from person to person, so you don't have to follow my settings.

The \LaTeX\ code of \autoref{tab:shorttitle} is shown below:
\begin{verbatim}
\begin{table}[!hbt]
\centering
\caption{A table with bold and center-aligned headers}
\label{tab:shorttitle}
\begin{tabularx}{0.5\textwidth}{crEE}
    \toprule
    \multirowthead{2}{Obs} & \multirowthead{2}{Attr 1} &
    \multicolumn{2}{c}{\thead{Attr 2}} \\
    \cmidrule(lr){3-4}
    & & \thead{SubAttr 1} & \thead{SubAttr 2} \\
    \midrule
    A & 1.0 & 2.0 & -12.0 \\
    B & -3.0 & -101.4 & 4.7 \\
    \bottomrule
\end{tabularx}
\end{table}   
\end{verbatim}

\begin{table}[!hbt]
    \centering
    \caption{A table with bold and center-aligned headers}\label{tab:shorttitle}
    \begin{tabularx}{0.5\textwidth}{crEE}
        \toprule
        \multirowthead{2}{Obs} & \multirowthead{2}{Attr 1} & \multicolumn{2}{c}{\thead{Attr 2}} \\
        \cmidrule(lr){3-4}
        & & \thead{SubAttr 1} & \thead{SubAttr 2} \\
        \midrule
        A & 1.0 & 2.0 & -12.0 \\
        B & -3.0 & -101.4 & 4.7 \\
        \bottomrule
    \end{tabularx}
\end{table}

We can refer to equations or floatings using \verb+\autoref+ commands, showing as \autoref{eq:exp}, \autoref{fig:longtitle} and \autoref{tab:shorttitle}. The bold format is achieved because the \verb+\autoref+ command has been redefined in the template.

\section{References Section}

The references section starts from a new page. TRB allows numeric citation or Harvard style citation. This article goes with a numeric citation style with round brackets. For example, I cite an article here.

\end{document}